\documentclass[journal,12pt,twocolumn]{IEEEtran}
\title{
Verification of First Distributive law of Boolean Algebra
}
\author{Beere Suresh}
\begin{document}
\maketitle
\begin{abstract}
This document shows the verification of first distributive law of Boolean Algerbra through Truth Table
\end{abstract}
\section{Statement}
This law states that 
X.(Y+Z) = X.Y + X.Z

This law can be verified by the Truth table mentioned below:
\begin{table}[h]
    	\centering
	\begin{tabular}{| c | c | c | c | c | c | c | c |}
	\hline
	\textbf{X} & \textbf{Y} & \textbf{Z} & \textbf{Y+Z} & \textbf{X.(Y+Z)} & \textbf{X.Y} & \textbf{X.Z} & \textbf{X.Y + X.Z} \\
	\hline
	0 & 0 & 0 & 0 & 0 & 0 & 0 & 0  \\
	 \hline
	 0 & 0 & 1 & 1 & 0 & 0 & 0 & 0  \\
	 \hline
	 0 & 1 & 0 & 1 & 0 & 0 & 0 & 0  \\
	 \hline
	0 & 1 & 1 & 1 & 0 & 0 & 0 & 0  \\
	\hline
	1 & 0 & 0 & 0 & 0 & 0 & 0 & 0  \\
	\hline
	1 & 0 & 1 & 1 & 1 & 0 & 1 & 1  \\
	\hline
	1 & 1 & 0 & 1 & 1 & 1 & 0 & 1  \\
	\hline
	1 & 1 & 1 & 1 & 1 & 1 & 1 & 1  \\
	\hline
	\end{tabular}
	\caption{1.1 Truth Table}
	\label{tab:my_label}
\end{table}
\section{Components}
\begin{table}[h]
	\centering
    	\begin{tabular}{| c | c | c |}
	\hline
	\textbf{Component}  &  \textbf{Value}  &  \textbf{Qunatity}\\
	\hline
	Arduino  & UNO & 1  \\
	\hline
	Jumper Wires  &  M-M  &  2  \\
	\hline
	BreadBoard  &    &  1\\
	\hline
	LED   &   &  1 \\
	\hline
\end{tabular}
\caption{1.1 Components}
\label{tab:my_label}
\end{table}
\section{Hardware}

\textbf{Problem 2.1}. Make connections between the Arduino UNO, and LED as shown in Table 2.1 \\


\begin{table}[h]
    	\centering
	\begin{tabular}{| c | c | c |}
	\hline
	Arduino & 12 & GND \\
	\hline
	LED & + ve & - ve \\
	\hline

	\hline
\end{tabular}
\caption{2.1 Connections}
\end{table}
\section{Software}
\textbf{Problem 3.1} Now connect the Arduino to the computer and execute the following program and verify the outputs as mentioned in Table 3.1 by modifying the inputs X, Y, Z.\\
\begin{table}[h]
	\raggedleft
	\begin{tabular}{| c |}
	\hline
	svn co https://github.com/sureshoye/IDE-Assignment/trunk/\\codes\\
	
	
	cd codes \\

	pio run \\

	pio run -t nobuild -t upload \\
    	
	\hline
        \end{tabular}

	\caption{}
 \end{table}
 
 \textbf{Result} You will observe that the light adjacent to Pin 13 and LED bulb glow together.
 \\
 \textbf{Problem 3.2} Now type "cd src" and open the programming code using the command "nano main.cpp" and modify the values of X,Y and Z in the programming code main.cpp file such that both LED and Light adjacent to Pin 13 turn off and then recompile it
\\
\end{document}
